\documentclass{article}
\usepackage{lmodern}
\usepackage{parcolumns}
\usepackage[margin=1cm]{geometry} 
\usepackage{graphicx} 
\graphicspath{{C:/Users/Owner/Desktop/Programing/LaTex/SciOly}}     
\usepackage{multicol}
\setlength{\columnsep}{.3cm}
\setlength{\columnseprule}{0pt}
\linespread{.8}
\begin{document}
\begin{multicols*}{3}{}
\begin{small}
\textbf{\underline{Units and Conversions}}
\end{small} 
\\
\fontsize{8pt}{8.25pt}\selectfont	
\textbf{Temperature}: Measure of average KE of particles in substance. $T_{C}=\frac{5}{9}(T_{F}-32);$ \\ $T_{F}=\frac{9}{5}T_{C} + 32; \: T_{K}= T_{C}+273.15; \: \\ T_{R}=T_{F}+459.67; \: T_{N}=\frac{1}{3}T_{C}; \: T_{Re}=\frac{4}{5}T_{C}; \\ T_{Ro}= \frac{21}{40}T_{C}+7.5; \:  T_{De}=\frac{3}{2}(100-T_{C})$ \\
	\textbf{Energy}: \underline{calorie}: amount of energy needed to raise 1 g of water by $1 ^{o}C$ (Cal is kcal); \underline{Joule}: 1 N applied over 1 m; \underline{BTU}: amount of energy needed to raise 1 lb of water by $1 ^{o}$F; \underline{Erg}: 1 dyn over 1 cm \\ \underline{Conversion factors}: 1 cal = .001 Cal = 4.184 J = .00397 BTU = $4.184$x$10^{7}$  ergs = .3238 ft$ \cdot$lb = $1.1628$x$10^{-6} kWh$ \\ **101.33 J= 1 L $\cdot$ atm*
\\
\textbf{Pressure}: 1 atm= 760 torr= 760 mm Hg = 101.325 kPa = 101325 Pa= 101325 N/$m^{2}$ 
\\
\textbf{Misc.}: $T_{C}=T_{F}$ at $-40^{o}$. 1 N = $10^{5}$ dyn. 1 planck=1.41679 x $10^{32}$ K.
\\
\begin{small}
		\textbf{\underline{Laws and Processes}}
\end{small}
\\
\textbf{Zeroth Law}: Two systems in equilibrium with a third are in equilibrium with each other.
\textbf{First Law}: $\Delta$ U = Q-W. The increase in internal energy of a closed system is equal to total of the energy added to the system (Heat supplied-work done). Total energy is conserved.
\textbf{Second Law}: Entropy of any closed system tends to increase; Carnot cycle is the upper limit on heat engine efficiency
\textbf{Third Law}: The entropy of a perfect crystal of any pure substance approaches zero as the temperature approaches absolute zero; The entropy of a system approaches a constant value as the temperature approaches zero.\\

\textbf{Carnot Cycle}: Isothermal exp, adiabatic exp, isothermal comp, adiabatic com; all reversible because $\Delta S$=0 (idealized). $\eta = W_{out}/Q_{in}$=$(T_{h}-T_{l})/T_{h}$
Where work = area under curve or $\int P dV$ \\
\underline{Isothermal}: Constant temperature, $\Delta$U=0 W=nRTln$(\frac{V_{2}}{V_{1}})$ \underline{Adiabatic}: Energy only transferred as work, no heat or mass transfer W=n$C_{v}(T_{l}-T_{h})$= $\frac{K(V_{f}^{1-\gamma}-V_{i}^{1-\gamma})}{1-\gamma}$ where $K=PV^{\gamma}$ and $\gamma=\frac{C_{p}}{C_{v}}$. \underline{Isochoric}: Constant volume W=0. \underline{Isobaric}: Constant pressure W=P$\Delta V$ \\

\textbf{Otto Cycle}: r. adiabatic exp, isochoric, r. adiabatic comp, isochoric. $\eta = 1 - \frac{1}{(1-r^{({\gamma}-1)})}$ = 1-$\frac{(T_{4}-T_{3})}{(T_{2}-T_{1})}$. \textbf{Refrigeration}: Reversed heat engine (uses work to move heat from colder region to hot region). $K_{cop}$=$\frac{Q_{L}}{W}$. $\eta =  \frac{T_{l}}{T_{H}-T_{L}}$ \\
\begin{small}
\textbf{\underline{State Functions and Chem :)}}
\end{small} \\
\textbf{Enthalpy}: Denoted by $\Delta$H=U+pV. Heat content of a system. Endothermic: $\Delta$H$>$0. Exothermic $\Delta$H$<$0. Increasing T favors endo process. \underline{Hess's Law}: Enthalpys are additive and multiplicative when combining reactions. \textbf{Entropy}: Denoted by $\Delta$S. Randomness of a system. Entropy of ions arbitrarily compared to $H^{+}$ which is set to 0 as reference. For ideal gas $\Delta$S= $C_{p}$ln$\frac{T_{2}}{T_{1}}$-nRln$\frac{p_{2}}{p_{1}}$. S=klogW (engraved on Boltzmann's grave! and W is number of microstates of the gas). dS=$\frac{dQ}{T}$. $\Delta S_{surroundings}$= - $\frac{\Delta H}{T}$. \textbf{Gibbs Free Energy}: Denoted by $\Delta$G=$\Delta$H-T$\Delta$S. Max amount of work that can be done by system. Spontaneous if $\Delta$G is negative, nonspont if positive, equilibrium if zero. G=U-TS+PV. \textbf{Helmholtz Free Energy}: Denoted by F=U-TS. \textbf{Degrees of Freedom}: Intensive variables that must be known to define the system completely. Eg. water-steam system has one degree of freedom because P and T are only variables, knowing P fixes T and vice versa. \\
\begin{small}
\textbf{\underline{Heat Transfer}}
\end{small}
\\
Q=mc$\Delta$T; $Q_{latent}$=mL; $\frac{Q}{t}$=$\frac{kA \Delta T}{x}$; Thermal Expansions: 1D- $\Delta$L=k$L_{o}$ $\Delta$T; 2D- $\Delta$A=2k$A_{o}$ $\Delta$T; 3D- $\Delta$V=3k$V_{o}$ $\Delta$T. \\
\begin{small}
\textbf{\underline{Constants!!!}}
\end{small}
\\
\textbf{Specific Heats} ($\frac{J}{g^{o}C}$): Ice: 2.108; Water: 4.186; Steam: 2.010. \textbf{Latent Heats for Water}($\frac{J}{g}$): $\Delta H_{f}$=334; $\Delta H_{v}$=2,230. \textbf{Ideal Gas Constant}: 8.314 $\frac{J}{\overline{m}K}$= .0821$\frac{L atm}{\overline{m}K}$; \textbf{Boltzmann Constant}: 1.3806 J/K; \textbf{Avogadro's Number}: $6.022x10^{23}$ units/$\overline{m}$; \textbf{Stefan-Boltzmann Constant}: $\sigma$=5.67x$10^{8}$ $\frac{W}{m^{2}K^{4}}$; \textbf{Planck's Constant}: 6.63x$10^{-34}$ Js; $\gamma$ is 5/3 for ideal monatomic gas, 1.4 for diatomic and air.
\\
\begin{Small}
\textbf{\underline{Phase Changes and Important Temperatures}}
\end{Small}
\\

\textbf{Phases} Solid, Liquid, Gas, Plasma, Bose-Einstein Condensate \textbf{Phase Changes}: Solid to Liquid: Melting; Liquid to Solid: Freezing; Liquid to Gas: Boiling; Gas to Liquid: Condensation; Solid to Gas: Sublimation; Gas to Solid: Deposition. \textbf{Random Temperatures $(^{o}C)$}: Lowest recorded surface temp:-89; Avg surface temp:15; Highest surface temp:58; Fahrenheit's Ice/Salt Mixture:-17.78; Surface of sun:5526
\\
\begin{small}
\textbf{\underline{Kinetic Theory}}
\end{small}
\\
Ideal gases composed of particles with negligible   volume in constant random motion, no intermolecular forces (straight line paths), all collisions are perfectly elastiic, average KE solely dependent on temperature.\\

\\ \textbf{Maxwell-Boltzmann distribution}: describes the probability of a particle in a given ideal sample having a speed (flattens at high T). \textbf{Formulas}: pV=nRT; mean speed= $\langle v \rangle$ = $\sqrt{\frac{8k_{B}T}{\pi m}}$; rms=$\sqrt{\langle} v^{2} \rangle$ = $\sqrt{\frac{3k_{B}T}{m}}$; most probable speed=$\sqrt{\frac{2k_{B}T}{m}}$; $KE_{avg}=\frac{3}{2}k_{B}T$; ln$\frac{P_{2}}{P_{1}}$=$\frac{-\Delta H}{R}(\frac{1}{T_{2}}-\frac{1}{T_{1}})$
\\
\begin{small}
\textbf{\underline{Blackbody Radiation}}
\end{small}
\\
An object that absorbs all radiation falling on it, at all wavelengths, is called a black body. Emission has a characteristic frequency distribution that depends on temperature.  $\lambda_{max}T=2.898x10^{-3} mK$; R=$\sigma T^{4}$ ($\sigma$ is Stefan Boltzmann Constant); P=$\varepsilon \sigma$A($T_{4}-T_{surroundings}^{4})$
\begin{small}
\textbf{\underline{History (Condensed)}}
\end{small}
\\
\textbf{William Thomson (1st Baron Kelvin)}:
Scottish/Irish;
2nd law of thermo w/ Clausius, Kelvin scale, dynamical theory of heat transfer, mathematical analysis of electricity, magnetism
\textbf{James Prescott Joule}:
English;
P=$I^2$R. Related work to heat in \emph{On the Mechanical Equivalent of Heat} (Paddle Wheel Experiment). Helped describe the first law of thermodynamics. Collaborated with Kelvin on his temperature scale. Joule-Thomson effect (when gas expands without producing work temperature falls). Won Copley Medal of the Royal Society. Joule unit of energy named after him.
\textbf{Benjamin Thompson (Count Rumford)}:
American Born, British;
Rejected the belief that heat is a form of matter and began modern theory that heat is a form of motion. Designed the Rumford photometer measure light intensity.
\textbf{Hermann von Helmholtz}:
German;
Contributions to conservation of energy. Field theory
\textbf{Daniel Bernoulli}
Swiss;
Fluid dynamics (Famous Bernoulli equation) as a result of conservation of energy
\textbf{Benoit Paul Emile Clapeyron}:
French;
Clausius Clapeyron equation (relates vapor pressure to temperature). Refined Carnot cycle using analytic methods. Graphed Carnot cycle on a closed PV curve. Contributed to second law of thermodynamics. Famous publication: memoir on the motive power of heat. Put together ideal gas law.
\textbf{Constantin Caratheodory}:
Greek, spent most of his life in Germany;
Published \emph{Investigations on the Foundations of Thermodynamics}
\textbf{Rudolf Diesel}:
German;
Invented the diesel engine
\textbf{Daniel Gabriel Fahrenheit}:
Dutch-German-Polish;
Invented mercury thermometer, fahrenheit temperature scale (with 0 F-100 F being the temperature range in Europe).
\textbf{Anders Celsius}:
Swedish;
Created celsius temperature scale
\textbf{Rudolf Clausius}:
German;
Famous statement of the second law of thermodynamics. Published \emph{On the Moving Force of Heat} which refined the theory of heat and the second law. Introduced entropy and virial theorem of heat. Contributed to kinetic theory; introduced mean free path between gas molecules.
\textbf{Ludwig Boltzmann}:
Austrian;
Tombstone has formula S=klogW inscribed. Used statistical means to contribute to kinetic theory of gasses (Maxwell Boltzmann distribution). 
Boltzmann Constant k=1.38x$10^{23}$ J/K
\textbf{Walther Nernst}:
German;
Stated New Heat Theorem which would later be termed third law of thermodynamics. Study thermodynamics of electrical currents through solutions. Nernst equation: cell potential based on gibbs free energy,
\textbf{Nicolas Leonard Sadi Carnot}:
French;
Published Reflections on the Motive Power of Fire
Famous for Carnot Cycle, Carnot Efficiency, Carnot Theorem
Carnot's Principle - Historical Origin of the 2nd Law
\textbf{Josiah Willard Gibbs}:
American;
Worked with Maxwell and Boltzmann on statistical mechanics. Received first American doctorate in engineering. 
\textbf{Johannes Diderik van der Waals}:
Dutch;
Known for van der Waals equation of state, Real Gas Law, van der Waals' forces
\textbf{James Clerk Maxwell}:
Scottish;
Maxwell's Equations, Maxwell-Boltzmann Distribution, Maxwell's Daemon
\textbf{John Herapath}:
English;
Gave a partial account of the kinetic theory of gases
\textbf{John Smeaton}:
British engineer; 
Improved steam engines
\textbf{Joseph Black}:
Scottish;
Discovered specific and latent heat
\textbf{Max Planck}:
German;
Famous for black body radiation, planck's constant, discovering energy quanta.


\end{multicols*}
\end{document}