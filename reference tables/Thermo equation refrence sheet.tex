\documentclass[8pt]{extarticle}
\pagestyle{empty}
\usepackage[margin=0.15in]{geometry}
\usepackage{amsmath}
\usepackage{multicol}
\usepackage{mathtools}
	\DeclarePairedDelimiter\abs{\lvert}{\rvert}
	\DeclarePairedDelimiter\norm{\lVert}{\rVert}

\begin{document}
\begin{small}
\section{Constants}
Universal Gas Constant $R=8.314 \frac{J}{mol \cdot K}$,$R=0.821 \frac{L\cdot atm}{mol \cdot K}$ \hspace{0.1in} Boltzmann Constant $k=1.38064852\times10^{-23}  \frac{J}{K}$ \hspace{0.1in} Avogadro's Number $N_A=6.022\times10^{23} \frac{units}{mol}$ \\
Stefan-Boltzmann Constant $\sigma=5.67\times10^8\frac{W}{m^2K^4}$ \hspace{0.1in} Planck's Constant $6.63\times10^{-34}J\cdot S$ \hspace{.1in} Planck Temperature $T_P=\dfrac{m_Pc^2}{k}=\sqrt{\dfrac{\hbar c^5}{Gk^2}}\approx1.416808\times10^{32}K$\\ \textbf{Specific Heats} ($\frac{J}{g^{o}C}$) Water - 4.186; Ice - 2.108; Steam - 2.010; Gold - .129; Lead - .129; Aluminum - .897; Steel - .490; Liquid Ammonia - 4.700; Gaseous Ammonia - 2.060 \textbf{Latent Heats}($\frac{J}{g}$) Water: $L_f=334$, $L_v=2,258$; Aluminum: $L_f=397$, $L_v=10,900$; Silicon: $L_f=1790$, $L_v=12,800$ \hspace{.1in} $\gamma$=5/3 for ideal monatomic gas, 1.4 for diatomic gas and air. 

\subsection{Conversions and Temperature}
 $T_{C}=\frac{5}{9}(T_{F}-32); \hspace{0.1in} T_{F}=\frac{9}{5}T_{C} + 32;\hspace{0.1in} T_{K}= T_{C}+273.15\hspace{0.1in} T_{R}=T_{F}+459.67;\hspace{0.1in} T_{N}=\frac{1}{3}T_{C};\hspace{0.1in} T_{Re}=\frac{4}{5}T_{C};\hspace{0.1in} T_{Ro}= \frac{21}{40}T_{C}+7.5; \hspace{0.1in} T_{De}=\frac{3}{2}(100-T_{C})$\\
1 cal = .001 Cal = 4.184 J = .00397 BTU = $4.184\times10^{7}$  ergs = .3238 ft$ \cdot$lb = $1.1628\times10^{-6} kWh$ \hspace{.1in} **101.33 J= 1 L $\cdot$ atm**\\
1 atm= 760 torr= 760 mm Hg = 101.325 kPa = 101325 Pa= 101325 $\frac{N}{m^2}$

\section{General} 
First Law $\Delta U=Q-W$ \hspace{.1in} Thermodynamic Temperature $\tau=kT$ \hspace{.1in} Thermodynamic Beta, Inverse Temperature $\beta=\dfrac{1}{kT}$ \hspace{.1in}  Fundamental thermodynamic relation $dU=TdS-PdV+\sum_{i=1}^{k}\mu_idN_i$\hspace{.1in} Internal Energy of a substance: $\Delta U=NC_V\Delta T$\hspace{.1in} Meyer's Equation $C_P-C_V=nR$

\section{Heat Engines and Refrigeration}
efficiency $\eta=\abs*{\dfrac{W}{Q_H}}$ \hspace{.1in} Carnot Efficiency $\eta_c=1-\abs*{\dfrac{Q_L}{Q_H}}=1-\dfrac{T_L}{T_H}$\\
Refrigeration coefficient of performance $K=\abs*{\dfrac{Q_L}{W}}$ \hspace{.1in} Carnot Refrigerator $K_C=\dfrac{\abs*{Q_L}}{\abs*{Q_H}-\abs*{Q_L}}=\dfrac{T_L}{T_H-T_L}$\\
$W_C=Q_H-Q_C=\bigg(1-\dfrac{T_C}{T_H}\bigg)Q_H$ - Carnot work, also equal to total curve area\\
Otto Cycle: $\eta=1-\dfrac{1}{1-r^{(\gamma-1)}}=1-\dfrac{T_4-T_3}{T_2-T_1}$

\section{Heat Transfer}Heat to raise temp $Q=mC\Delta T$ \hspace{.1in} Latent heat $Q=mL$ \hspace{.1in} heat transfer $q=\dfrac{Q}{t}=\dfrac{kA\Delta T}{d}$ \hspace{.1in} linear thermal expansion: $\Delta L=\alpha L_0\Delta T$, Area: $\Delta A=2A_0\alpha\Delta T$, Volume: $\Delta V=3V_0\alpha\Delta T$ \hspace{.1in} coefficient of thermal expansion $\alpha_p=-\dfrac{1}{V}\bigg(\dfrac{\partial V}{\partial P}\bigg)_{T,N}$ Thermal conduction rate, thermal current, thermal/heat flux, thermal power transfer $P=\dfrac{dQ}{dt}$, Thermal Intensity $I=\dfrac{dP}{dA}$, Thermal/heat flux density (vector analogue of thermal intensity above) (symbol is little $q$) $Q=\iint q\cdot dSdt$ \hspace{.1in} $C_P=\bigg(\dfrac{\partial Q_{rev}}{\partial T}\bigg)_P=\bigg(\dfrac{\partial U}{\partial T}\bigg)_P+P\bigg(\dfrac{\partial V}{\partial T}\bigg)=\bigg(\dfrac{\partial H}{\partial T}\bigg)_P=T\bigg(\dfrac{\partial S}{\partial T}\bigg)_P$\hspace{.1in}
 $C_V=\bigg(\dfrac{\partial Q_{rev}}{\partial T}\bigg)_V=\bigg(\dfrac{\partial U}{\partial T}\bigg)_V=T\bigg(\dfrac{\partial S}{\partial T}\bigg)_V$, Blackbody Radiation: $I=\dfrac{P}{A}=\sigma T^4$, Real bodies: $I=\dfrac{P}{A}=\epsilon\sigma T^4$
\section{Gas Laws}
$\dfrac{P_1V_1}{T_1}=\dfrac{P_2V_2}{T_2}$ - Combined Gas Law\hspace{0.1in}
$P_1V_1=P_2V_2$ - Boyle's Law \hspace{0.1in}
$\dfrac{V_1}{T_1}=\dfrac{V_2}{T_2}$ - Charles' Law\hspace{0.1in}
$\dfrac{P_1}{T_1}=\dfrac{P_2}{T_2}$ - Gay-Lussac's Law\\
$(P+\dfrac{n^2a}{V^2})(V-nb)=nRT$ - van der Waals' Equation (real gas law)\hspace{.1in}
$\ln\dfrac{P_2}{P_1}=-\dfrac{\Delta H_{vap}}{R}\bigg(\dfrac{1}{T_2}-\dfrac{1}{T_1}\bigg)$ - Clausius Clapeyron Equation \hspace{.1in}
$\gamma=\dfrac{C_p}{C_V}$ Adiabatic Index

\subsection{Ideal Gases}
Ideal Gas Law $PV=nRT=NkT$, $\dfrac{P_1V_1}{P_2V_2}=\dfrac{n_1T_1}{n_2T_2}=\dfrac{N_1T_1}{N_2T_2}$ \hspace{.1in}
Pressure $P=\dfrac{Nm\langle v^2\rangle}{3V}=\dfrac{nM_m\langle v^2\rangle}{3V}=\frac{1}{3}\rho\langle v^2\rangle$\\
\begin{tabular}{ | p{1.5cm} | p{4cm} | p{1.5cm} | p{2cm} | p{1.5cm} | p{4.7cm} | }
\hline
Quantity & General & Isobaric & Isochoric & Isothermal & Adiabatic \\
\hline
Work $W$ & $\delta W=-pdV$ & $-p\Delta V$ & $0$ & $-nRT\ln\dfrac{V_2}{V_1}$ or $-nRT\ln\dfrac{P_1}{P_2}$ & \vspace{.01in} $\dfrac{PV^\gamma(V^{1-\gamma}_f-V^{1-\gamma}_i)}{1-\gamma}=C_V(T_2-T_1)$ \\
\hline
Heat Capacity $C$ & (same as real gas) & \vspace{.001in} $C_p=\frac{5}{2}nR$ (monatomic) or $C_p=\frac{7}{2}nR$ (diatomic) & \vspace{.001in} $C_V=\frac{3}{2}nR$ (monatomic) or $C_V=\frac{5}{2}nR$ (diatomic)& & \\
\hline
Internal Energy $\Delta U$ & $\Delta U=C_V\Delta T$ & $Q+W$ or $Q_p-p\Delta V$ & $Q$ or \hspace{.1in} $C_V(T_2-T_1)$ & $0$ or $Q=-W$ & $W$ or $C_V(T_2-T_1)$ \\
\hline
Enthalpy $\Delta H$ & $H=U+PV$ & $C_p(T_2-T_1)$ & $Q_V + V\Delta p$ & $0$ & $C_p(T_2-T_1)$ \\
\hline
Entropy $\Delta S$ & $\Delta S=C_V\ln \dfrac{T_2}{T_1}+nR\ln\dfrac{V_2}{V_1}$ $\Delta S=C_p\ln\dfrac{T_2}{T_1}-nR\ln\dfrac{P_2}{P_1}$ & $C_p\ln\dfrac{T_2}{T_1}$ & $C_V\ln\dfrac{T_2}{T_1}$ & $nR\ln\dfrac{V_2}{V_1}$ $\dfrac{Q}{T}$ & $C_p\ln\dfrac{V_2}{V_1}+C_V\ln\dfrac{P_2}{P1}=0$ \\
\hline
constant & & $\dfrac{V}{T}$ & $\dfrac{P}{T}$ & $PV$ & $PV^\gamma$ \\
\hline
\end{tabular}

\subsection{Statistical Thermodynamics}
mean speed $\langle v\rangle=\sqrt{\dfrac{8kT}{\pi m}}$ \hspace{.1in} Root mean square speed $v_{rms}=\sqrt{\langle v^2\rangle}=\sqrt{\dfrac{3kT}{m}}$ \hspace{.1in} most probable speed $v_{mode}=\sqrt{\dfrac{2kT}{m}}$ \hspace{.1in} Mean free path $\lambda=\dfrac{RT}{\sqrt{2}\pi d^2N_AP}$ \\
avg. KE per molecule $KE_{avg}=\frac{3}{2}kT$, per mole: $KE_{avg}=\frac{3}{2}RT$

\pagebreak
\section{Chem}
$\Delta H=U+PV$ - Enthalpy \hspace{0.1in}
$\Delta_rH^\Theta=\Sigma vH^\Theta_{products}-\Sigma vH^\Theta_{reactants}$ - Standard Enthalpy of Formation\\
$S=k\log W$, $dS=\dfrac{dQ}{T}$, $\Delta S_{surroundings}=-\dfrac{\Delta H}{T}$ - Entropy \hspace{0.1in} $\Delta G=\Delta H-T\Delta S$, $G=U+PV-TS$ - Gibbs Free Energy \hspace{.1in} \textbf{Other Potentials} Helmholtz Free Energy $F=U-TS$, Landau potential (Landau Free Energy, Grand Potential) $\Omega=U-TS-\mu N$ ($\mu=$ chemical potential), Massieu Potential, Helmholtz free entropy $\Phi=S-\dfrac{U}{T}$, Planck potential, Gibbs free entropy $\Xi=\Phi-\dfrac{PV}{T}$

\begin{multicols*}{2}
\section{processes}
\begin{tabular}{| p{2cm} | p{3cm} |}
\hline
Process & Equations\\
\hline
Isentropic (Adiabatic Reversible) & $\Delta Q=0$, $\Delta U=-\Delta W$ For an ideal gas: $P_1V_1^\gamma=P_2V_2^\gamma$, $T_1V_1^{\gamma-1}=T_1V_2^{\gamma-1}$, $P_1^{\gamma-a}T_1^\gamma=P_2^{\gamma-1}T_2^\gamma$ \\
\hline
Isothermal & $\Delta U=0$, $\Delta W=\Delta Q$, For an ideal gas: $W=kTN\ln\dfrac{V_2}{V_1}$\\
\hline
Isobaric & $P_1=P_2$, $P=$constant, $\Delta W=P\Delta V$, $\Delta q=\Delta U + P\delta V$ \\
\hline
Isochoric & $V_1=V_2$, $V=$constant, $\Delta W=0$, $\Delta Q=\Delta U$\\
\hline
Free expansion & $\Delta U=0$\\
\hline
Work Done by expanding a gas & Process: $$\Delta W=\int_{V_1}^{V_2} PdV$$, Net work in Cyclic processes: $$\Delta W=\oint_{cycle} PdV$$ \\
\hline
\end{tabular}


\section{Quantum Properties}
$U=NkT^2\bigg(\dfrac{\partial\ln Z}{\partial T}\bigg)_V$ \hspace{.1in} $S=\dfrac{U}{T}+N$, For indistinguishable particles: $S=\dfrac{U}{T}+Nk\ln N+Nk$\\
\begin{tabular}{| p{2cm} | p{3cm} |}
\hline
Degree of freedom & Partition Function \\
\hline
Translation & $Z_t=\dfrac{(2\pi mkT)^{\frac{3}{2}}V}{h^3}$\\
\hline
Vibration & $Z_v=\dfrac{1}{1-e^{(\frac{-h\omega}{2\pi kT})}}$\\
\hline
Rotation & $Z_r=\dfrac{2IkT}{\sigma\hbar^2}$, $\sigma$=1 for heteronuclear molecules and $\sigma$=2 for homonuclear. \\
\hline
\end{tabular}
\section{other}
\subsection{electricity}
resistors - $Q=I^2Rt=tIV=\dfrac{tV^2}{R}$
\end{multicols*}
\end{small}
\end{document}
